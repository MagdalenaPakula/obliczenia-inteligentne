\documentclass[10pt]{article}

\usepackage{tabularx}
\usepackage[a4paper,margin=2.5cm, bottom=2.5cm]{geometry}
\usepackage{fancyhdr}
\usepackage{listings}
\usepackage{booktabs}
\usepackage{float}
\usepackage{subcaption}
% \usepackage{caption}
% \captionsetup{font=footnotesize}
\usepackage{graphicx}
\usepackage{amsmath}
\usepackage{amssymb}
\usepackage{amsthm}
\usepackage{array}
\usepackage[table]{xcolor}
\usepackage{pgfplots}
\pgfplotsset{compat=1.17}
\usepackage{pgfplotstable}
\usepackage{multirow}
\usepackage{tikz}
\usepackage[hidelinks]{hyperref}
\usepackage{titling}
\usepackage[T1]{fontenc} % To ensure correct hyphenation and character encoding
\usepackage[polish]{babel} % Polish language support

\setlength{\headheight}{40pt}
\setlength{\parindent}{0pt}
\setlength{\parskip}{1ex}
\renewcommand{\headrulewidth}{0pt}

\pagestyle{fancy}
\fancyhead{}
\fancyhead[L]{
    \renewcommand{\arraystretch}{1.5}
    \begin{tabularx}{\textwidth}{|X|X|}
        \hline
        \bfseries Obliczenia inteligentne & \bfseries \thetitle \\
        \hline
    \end{tabularx}
}
\fancyfoot[C]{\thepage}

\renewcommand{\maketitle}{
    \thispagestyle{plain}
    \renewcommand{\arraystretch}{2}
    \vspace*{-8em}
    \footnotesize
    \begin{flushleft}
        \begin{tabularx}{\textwidth}{|X|X|}
            \hline
            \bfseries Obliczenia Inteligentne  & \bfseries \thetitle                           \\ \hline
            \multicolumn{2}{|l|}{
                \begin{tabular}[t]{@{}ll@{}} 
                    \textbf{Grupa:} Grupa 1
                    \hspace{4.5em}
                    \textbf{Dzień i czas:} Czwartek, 10:00
                    \hspace{4.5em}
                    \textbf{Rok akademicki:} 2023/24
                \end{tabular}
            } \\ \hline
            \multicolumn{2}{|l|}{
                \begin{tabular}[t]{@{}l@{\hspace{10em}}l@{}} 
                    \textbf{Imię i nazwisko:} \textsc{Jakub Pawlak} & \textbf{Imię i nazwisko:} \textsc{Magdalena Paku\l a} 
                \end{tabular}
            } \\
            \hline
        \end{tabularx}
    \end{flushleft}
    \renewcommand{\arraystretch}{1}
}


\title{Projekt 2 --- Zadanie 2}
\captionsetup{font=small}
\begin{document}

\maketitle
\normalsize

\section{Eksperyment 1: Architektury sieci splotowej dla MNIST (Jakub Pawlak)}\label{sec:ex1-pawlak_mnist}

\paragraph{Pierwsza architektura}
\paragraph{Druga architektura prowadząca do ekstrakcji 2 cech}

\pagebreak
\section{Eksperyment 1: Architektury sieci splotowej dla MNIST (Magdalena Pakuła)}\label{sec:ex1-pakula_mnist}

\paragraph{Pierwsza architektura}
\paragraph{Druga architektura prowadząca do ekstrakcji 2 cech}

\pagebreak
\section{Eksperyment 1: Architektury sieci splotowej dla CIFAR10 (Jakub Pawlak)}\label{sec:ex1-pawlak_cifar}

\paragraph{Pierwsza architektura}
\paragraph{Druga architektura prowadząca do ekstrakcji 2 cech}

\pagebreak
\section{Eksperyment 1: Architektury sieci splotowej dla CIFAR10 (Magdalena Pakuła)}\label{sec:ex1-pakula_cifar}

\paragraph{Pierwsza architektura}
\paragraph{Druga architektura prowadząca do ekstrakcji 2 cech}

\pagebreak
\section{Eksperyment 2: Wyniki dla MNIST}\label{sec:ex2_mnist}

\paragraph{Najlepsza architektura}
\paragraph{Najelpsza architektura prowadząca do ekstrakcji 2 cech}

\pagebreak
\section{Eksperyment 2: Wyniki dla CIFAR10}\label{sec:ex2_cifar}

\paragraph{Najlepsza architektura}
\paragraph{Najelpsza architektura prowadząca do ekstrakcji 2 cech}

\pagebreak
\section{Analiza i wnioski}\label{sec:wyniki}

\paragraph{Porównanie architektur sieci splotowych}

\paragraph{Wpływ augmentacji danych}

\end{document}