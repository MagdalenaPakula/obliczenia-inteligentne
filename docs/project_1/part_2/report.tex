\documentclass[12pt]{article}

\usepackage{tabularx}
\usepackage[a4paper,margin=2.5cm, bottom=2.5cm]{geometry}
\usepackage{fancyhdr}
\usepackage{listings}
\usepackage{booktabs}
\usepackage{float}
\usepackage{subcaption}
% \usepackage{caption}
% \captionsetup{font=footnotesize}
\usepackage{graphicx}
\usepackage{amsmath}
\usepackage{amssymb}
\usepackage{amsthm}
\usepackage{array}
\usepackage[table]{xcolor}
\usepackage{pgfplots}
\pgfplotsset{compat=1.17}
\usepackage{pgfplotstable}
\usepackage{multirow}
\usepackage{tikz}
\usepackage[hidelinks]{hyperref}
\usepackage{titling}
\usepackage[T1]{fontenc} % To ensure correct hyphenation and character encoding
\usepackage[polish]{babel} % Polish language support

\setlength{\headheight}{40pt}
\setlength{\parindent}{0pt}
\setlength{\parskip}{1ex}
\renewcommand{\headrulewidth}{0pt}

\pagestyle{fancy}
\fancyhead{}
\fancyhead[L]{
    \renewcommand{\arraystretch}{1.5}
    \begin{tabularx}{\textwidth}{|X|X|}
        \hline
        \bfseries Obliczenia inteligentne & \bfseries \thetitle \\
        \hline
    \end{tabularx}
}
\fancyfoot[C]{\thepage}

\renewcommand{\maketitle}{
    \thispagestyle{plain}
    \renewcommand{\arraystretch}{2}
    \vspace*{-8em}
    \footnotesize
    \begin{flushleft}
        \begin{tabularx}{\textwidth}{|X|X|}
            \hline
            \bfseries Obliczenia Inteligentne  & \bfseries \thetitle                           \\ \hline
            \multicolumn{2}{|l|}{
                \begin{tabular}[t]{@{}ll@{}} 
                    \textbf{Grupa:} Grupa 1
                    \hspace{4.5em}
                    \textbf{Dzień i czas:} Czwartek, 10:00
                    \hspace{4.5em}
                    \textbf{Rok akademicki:} 2023/24
                \end{tabular}
            } \\ \hline
            \multicolumn{2}{|l|}{
                \begin{tabular}[t]{@{}l@{\hspace{10em}}l@{}} 
                    \textbf{Imię i nazwisko:} \textsc{Jakub Pawlak} & \textbf{Imię i nazwisko:} \textsc{Magdalena Paku\l a} 
                \end{tabular}
            } \\
            \hline
        \end{tabularx}
    \end{flushleft}
    \renewcommand{\arraystretch}{1}
}


\usepackage{pgfplotstable}
\title{Projekt 1 - Zadanie 2}

\newcommand*{\subfigwidth}{0.24\textwidth}

\begin{document}
\maketitle

\begin{figure}[H]\centering
    \begin{subfigure}[t]{\subfigwidth}
        \includegraphics[width=\linewidth]{img/exp_1/mlp/2_1/identity/scores.png}
        \caption{identity 2\_1}
    \end{subfigure}
    \hfill
    \begin{subfigure}[t]{\subfigwidth}
        \includegraphics[width=\linewidth]{img/exp_1/mlp/2_1/identity/boundary.png}
        \caption{identity 2\_1}
    \end{subfigure}
    \hfill
    \begin{subfigure}[t]{\subfigwidth}
        \includegraphics[width=\linewidth]{img/exp_1/mlp/2_1/relu/scores.png}
        \caption{ReLu 2\_1}
    \end{subfigure}
    \hfill
    \begin{subfigure}[t]{\subfigwidth}
        \includegraphics[width=\linewidth]{img/exp_1/mlp/2_1/relu/boundary.png}
        \caption{ReLu 2\_1}
    \end{subfigure}
    \\
    \begin{subfigure}[t]{\subfigwidth}
        \includegraphics[width=\linewidth]{img/exp_1/mlp/2_2/identity/scores.png}
        \caption{identity 2\_2}
    \end{subfigure}
    \hfill
    \begin{subfigure}[t]{\subfigwidth}
        \includegraphics[width=\linewidth]{img/exp_1/mlp/2_2/identity/boundary.png}
        \caption{identity 2\_2}
    \end{subfigure}
    \hfill
    \begin{subfigure}[t]{\subfigwidth}
        \includegraphics[width=\linewidth]{img/exp_1/mlp/2_2/relu/scores.png}
        \caption{ReLu 2\_2}
    \end{subfigure}
    \hfill
    \begin{subfigure}[t]{\subfigwidth}
        \includegraphics[width=\linewidth]{img/exp_1/mlp/2_2/relu/boundary.png}
        \caption{ReLu 2\_2}
    \end{subfigure}
    \\
    \begin{subfigure}[t]{\subfigwidth}
        \includegraphics[width=\linewidth]{img/exp_1/mlp/2_3/identity/scores.png}
        \caption{identity 2\_3}
    \end{subfigure}
    \hfill
    \begin{subfigure}[t]{\subfigwidth}
        \includegraphics[width=\linewidth]{img/exp_1/mlp/2_3/identity/boundary.png}
        \caption{identity 2\_3}
    \end{subfigure}
    \hfill
    \begin{subfigure}[t]{\subfigwidth}
        \includegraphics[width=\linewidth]{img/exp_1/mlp/2_3/relu/scores.png}
        \caption{ReLu 2\_3}
    \end{subfigure}
    \hfill
    \begin{subfigure}[t]{\subfigwidth}
        \includegraphics[width=\linewidth]{img/exp_1/mlp/2_3/relu/boundary.png}
        \caption{ReLu 2\_3}
    \end{subfigure}
    \caption{Eksperyment 1 --- Perceptron MLP}
\end{figure}

\begin{figure}[H]\centering
    \begin{subfigure}[t]{\subfigwidth}
        \includegraphics[width=\linewidth]{img/exp_1/svm/2_1/linear/scores.png}
        \caption{linear 2\_1}
    \end{subfigure}
    \hfill
    \begin{subfigure}[t]{\subfigwidth}
        \includegraphics[width=\linewidth]{img/exp_1/svm/2_1/linear/boundary.png}
        \caption{linear 2\_1}
    \end{subfigure}
    \hfill
    \begin{subfigure}[t]{\subfigwidth}
        \includegraphics[width=\linewidth]{img/exp_1/svm/2_1/rbf/scores.png}
        \caption{rbf 2\_1}
    \end{subfigure}
    \hfill
    \begin{subfigure}[t]{\subfigwidth}
        \includegraphics[width=\linewidth]{img/exp_1/svm/2_1/rbf/boundary.png}
        \caption{rbf 2\_1}
    \end{subfigure}
    \\
    \begin{subfigure}[t]{\subfigwidth}
        \includegraphics[width=\linewidth]{img/exp_1/svm/2_2/linear/scores.png}
        \caption{linear 2\_2}
    \end{subfigure}
    \hfill
    \begin{subfigure}[t]{\subfigwidth}
        \includegraphics[width=\linewidth]{img/exp_1/svm/2_2/linear/boundary.png}
        \caption{linear 2\_2}
    \end{subfigure}
    \hfill
    \begin{subfigure}[t]{\subfigwidth}
        \includegraphics[width=\linewidth]{img/exp_1/svm/2_2/rbf/scores.png}
        \caption{rbf 2\_2}
    \end{subfigure}
    \hfill
    \begin{subfigure}[t]{\subfigwidth}
        \includegraphics[width=\linewidth]{img/exp_1/svm/2_2/rbf/boundary.png}
        \caption{rbf 2\_2}
    \end{subfigure}
    \\
    \begin{subfigure}[t]{\subfigwidth}
        \includegraphics[width=\linewidth]{img/exp_1/svm/2_3/linear/scores.png}
        \caption{linear 2\_3}
    \end{subfigure}
    \hfill
    \begin{subfigure}[t]{\subfigwidth}
        \includegraphics[width=\linewidth]{img/exp_1/svm/2_3/linear/boundary.png}
        \caption{linear 2\_3}
    \end{subfigure}
    \hfill
    \begin{subfigure}[t]{\subfigwidth}
        \includegraphics[width=\linewidth]{img/exp_1/svm/2_3/rbf/scores.png}
        \caption{rbf 2\_3}
    \end{subfigure}
    \hfill
    \begin{subfigure}[t]{\subfigwidth}
        \includegraphics[width=\linewidth]{img/exp_1/svm/2_3/rbf/boundary.png}
        \caption{rbf 2\_3}
    \end{subfigure}
    \caption{Eksperyment 1 --- SVM}
\end{figure}

\clearpage

\renewcommand*{\subfigwidth}{0.16\textwidth}
\begin{figure}[H]\centering
    \begin{subfigure}[t]{\subfigwidth}
        \includegraphics[width=\linewidth]{img/exp_2/knn/2_2/accuracy.png}
        \caption{Dokładność}
    \end{subfigure}
    \\
    \begin{subfigure}[t]{\subfigwidth}
        \includegraphics[width=\linewidth]{img/exp_2/knn/2_2/min/train_boundary.png}
        \caption{Min, train}
    \end{subfigure}
    \hfill
    \begin{subfigure}[t]{\subfigwidth}
        \includegraphics[width=\linewidth]{img/exp_2/knn/2_2/min/train_matrix.png}
        \caption{Min, train}
    \end{subfigure}
    \hfill
    \begin{subfigure}[t]{\subfigwidth}
        \includegraphics[width=\linewidth]{img/exp_2/knn/2_2/min/test_boundary.png}
        \caption{Min, test}
    \end{subfigure}
    \hfill
    \begin{subfigure}[t]{\subfigwidth}
        \includegraphics[width=\linewidth]{img/exp_2/knn/2_2/min/test_matrix.png}
        \caption{Min, test}
    \end{subfigure} 
    \\
    \begin{subfigure}[t]{\subfigwidth}
        \includegraphics[width=\linewidth]{img/exp_2/knn/2_2/best/train_boundary.png}
        \caption{Best, train}
    \end{subfigure}
    \hfill
    \begin{subfigure}[t]{\subfigwidth}
        \includegraphics[width=\linewidth]{img/exp_2/knn/2_2/best/train_matrix.png}
        \caption{Best, train}
    \end{subfigure}
    \hfill
    \begin{subfigure}[t]{\subfigwidth}
        \includegraphics[width=\linewidth]{img/exp_2/knn/2_2/best/test_boundary.png}
        \caption{Best, test}
    \end{subfigure}
    \hfill
    \begin{subfigure}[t]{\subfigwidth}
        \includegraphics[width=\linewidth]{img/exp_2/knn/2_2/best/test_matrix.png}
        \caption{Best, test}
    \end{subfigure} 
    \\
    \begin{subfigure}[t]{\subfigwidth}
        \includegraphics[width=\linewidth]{img/exp_2/knn/2_2/max/train_boundary.png}
        \caption{Max, train}
    \end{subfigure}
    \hfill
    \begin{subfigure}[t]{\subfigwidth}
        \includegraphics[width=\linewidth]{img/exp_2/knn/2_2/max/train_matrix.png}
        \caption{Max, train}
    \end{subfigure}
    \hfill
    \begin{subfigure}[t]{\subfigwidth}
        \includegraphics[width=\linewidth]{img/exp_2/knn/2_2/max/test_boundary.png}
        \caption{Max, test}
    \end{subfigure}
    \hfill
    \begin{subfigure}[t]{\subfigwidth}
        \includegraphics[width=\linewidth]{img/exp_2/knn/2_2/max/test_matrix.png}
        \caption{Max, test}
    \end{subfigure} 
    
    \caption{Eksperyment 2 --- KNN na zbiorze 2\_2}
\end{figure}

\begin{figure}[H]\centering
    \begin{subfigure}[t]{\subfigwidth}
        \includegraphics[width=\linewidth]{img/exp_2/knn/2_3/accuracy.png}
        \caption{Dokładność}
    \end{subfigure}
    \\
    \begin{subfigure}[t]{\subfigwidth}
        \includegraphics[width=\linewidth]{img/exp_2/knn/2_3/min/train_boundary.png}
        \caption{Min, train}
    \end{subfigure}
    \hfill
    \begin{subfigure}[t]{\subfigwidth}
        \includegraphics[width=\linewidth]{img/exp_2/knn/2_3/min/train_matrix.png}
        \caption{Min, train}
    \end{subfigure}
    \hfill
    \begin{subfigure}[t]{\subfigwidth}
        \includegraphics[width=\linewidth]{img/exp_2/knn/2_3/min/test_boundary.png}
        \caption{Min, test}
    \end{subfigure}
    \hfill
    \begin{subfigure}[t]{\subfigwidth}
        \includegraphics[width=\linewidth]{img/exp_2/knn/2_3/min/test_matrix.png}
        \caption{Min, test}
    \end{subfigure} 
    \\
    \begin{subfigure}[t]{\subfigwidth}
        \includegraphics[width=\linewidth]{img/exp_2/knn/2_3/best/train_boundary.png}
        \caption{Best, train}
    \end{subfigure}
    \hfill
    \begin{subfigure}[t]{\subfigwidth}
        \includegraphics[width=\linewidth]{img/exp_2/knn/2_3/best/train_matrix.png}
        \caption{Best, train}
    \end{subfigure}
    \hfill
    \begin{subfigure}[t]{\subfigwidth}
        \includegraphics[width=\linewidth]{img/exp_2/knn/2_3/best/test_boundary.png}
        \caption{Best, test}
    \end{subfigure}
    \hfill
    \begin{subfigure}[t]{\subfigwidth}
        \includegraphics[width=\linewidth]{img/exp_2/knn/2_3/best/test_matrix.png}
        \caption{Best, test}
    \end{subfigure} 
    \\
    \begin{subfigure}[t]{\subfigwidth}
        \includegraphics[width=\linewidth]{img/exp_2/knn/2_3/max/train_boundary.png}
        \caption{Max, train}
    \end{subfigure}
    \hfill
    \begin{subfigure}[t]{\subfigwidth}
        \includegraphics[width=\linewidth]{img/exp_2/knn/2_3/max/train_matrix.png}
        \caption{Max, train}
    \end{subfigure}
    \hfill
    \begin{subfigure}[t]{\subfigwidth}
        \includegraphics[width=\linewidth]{img/exp_2/knn/2_3/max/test_boundary.png}
        \caption{Max, test}
    \end{subfigure}
    \hfill
    \begin{subfigure}[t]{\subfigwidth}
        \includegraphics[width=\linewidth]{img/exp_2/knn/2_3/max/test_matrix.png}
        \caption{Max, test}
    \end{subfigure} 
    
    \caption{Eksperyment 2 --- KNN na zbiorze 2\_3}
\end{figure}

\clearpage

\begin{figure}[H]\centering
    \begin{subfigure}[t]{\subfigwidth}
        \includegraphics[width=\linewidth]{img/exp_2/svm/2_2/accuracy.png}
        \caption{Dokładność}
    \end{subfigure}
    \\
    \begin{subfigure}[t]{\subfigwidth}
        \includegraphics[width=\linewidth]{img/exp_2/svm/2_2/min/train_boundary.png}
        \caption{Min, train}
    \end{subfigure}
    \hfill
    \begin{subfigure}[t]{\subfigwidth}
        \includegraphics[width=\linewidth]{img/exp_2/svm/2_2/min/train_matrix.png}
        \caption{Min, train}
    \end{subfigure}
    \hfill
    \begin{subfigure}[t]{\subfigwidth}
        \includegraphics[width=\linewidth]{img/exp_2/svm/2_2/min/test_boundary.png}
        \caption{Min, test}
    \end{subfigure}
    \hfill
    \begin{subfigure}[t]{\subfigwidth}
        \includegraphics[width=\linewidth]{img/exp_2/svm/2_2/min/test_matrix.png}
        \caption{Min, test}
    \end{subfigure} 
    \\
    \begin{subfigure}[t]{\subfigwidth}
        \includegraphics[width=\linewidth]{img/exp_2/svm/2_2/best/train_boundary.png}
        \caption{Best, train}
    \end{subfigure}
    \hfill
    \begin{subfigure}[t]{\subfigwidth}
        \includegraphics[width=\linewidth]{img/exp_2/svm/2_2/best/train_matrix.png}
        \caption{Best, train}
    \end{subfigure}
    \hfill
    \begin{subfigure}[t]{\subfigwidth}
        \includegraphics[width=\linewidth]{img/exp_2/svm/2_2/best/test_boundary.png}
        \caption{Best, test}
    \end{subfigure}
    \hfill
    \begin{subfigure}[t]{\subfigwidth}
        \includegraphics[width=\linewidth]{img/exp_2/svm/2_2/best/test_matrix.png}
        \caption{Best, test}
    \end{subfigure} 
    \\
    \begin{subfigure}[t]{\subfigwidth}
        \includegraphics[width=\linewidth]{img/exp_2/svm/2_2/max/train_boundary.png}
        \caption{Max, train}
    \end{subfigure}
    \hfill
    \begin{subfigure}[t]{\subfigwidth}
        \includegraphics[width=\linewidth]{img/exp_2/svm/2_2/max/train_matrix.png}
        \caption{Max, train}
    \end{subfigure}
    \hfill
    \begin{subfigure}[t]{\subfigwidth}
        \includegraphics[width=\linewidth]{img/exp_2/svm/2_2/max/test_boundary.png}
        \caption{Max, test}
    \end{subfigure}
    \hfill
    \begin{subfigure}[t]{\subfigwidth}
        \includegraphics[width=\linewidth]{img/exp_2/svm/2_2/max/test_matrix.png}
        \caption{Max, test}
    \end{subfigure} 
    
    \caption{Eksperyment 2 --- SVM na zbiorze 2\_2}
\end{figure}

\begin{figure}[H]\centering
    \begin{subfigure}[t]{\subfigwidth}
        \includegraphics[width=\linewidth]{img/exp_2/svm/2_3/accuracy.png}
        \caption{Dokładność}
    \end{subfigure}
    \\
    \begin{subfigure}[t]{\subfigwidth}
        \includegraphics[width=\linewidth]{img/exp_2/svm/2_3/min/train_boundary.png}
        \caption{Min, train}
    \end{subfigure}
    \hfill
    \begin{subfigure}[t]{\subfigwidth}
        \includegraphics[width=\linewidth]{img/exp_2/svm/2_3/min/train_matrix.png}
        \caption{Min, train}
    \end{subfigure}
    \hfill
    \begin{subfigure}[t]{\subfigwidth}
        \includegraphics[width=\linewidth]{img/exp_2/svm/2_3/min/test_boundary.png}
        \caption{Min, test}
    \end{subfigure}
    \hfill
    \begin{subfigure}[t]{\subfigwidth}
        \includegraphics[width=\linewidth]{img/exp_2/svm/2_3/min/test_matrix.png}
        \caption{Min, test}
    \end{subfigure} 
    \\
    \begin{subfigure}[t]{\subfigwidth}
        \includegraphics[width=\linewidth]{img/exp_2/svm/2_3/best/train_boundary.png}
        \caption{Best, train}
    \end{subfigure}
    \hfill
    \begin{subfigure}[t]{\subfigwidth}
        \includegraphics[width=\linewidth]{img/exp_2/svm/2_3/best/train_matrix.png}
        \caption{Best, train}
    \end{subfigure}
    \hfill
    \begin{subfigure}[t]{\subfigwidth}
        \includegraphics[width=\linewidth]{img/exp_2/svm/2_3/best/test_boundary.png}
        \caption{Best, test}
    \end{subfigure}
    \hfill
    \begin{subfigure}[t]{\subfigwidth}
        \includegraphics[width=\linewidth]{img/exp_2/svm/2_3/best/test_matrix.png}
        \caption{Best, test}
    \end{subfigure} 
    \\
    \begin{subfigure}[t]{\subfigwidth}
        \includegraphics[width=\linewidth]{img/exp_2/svm/2_3/max/train_boundary.png}
        \caption{Max, train}
    \end{subfigure}
    \hfill
    \begin{subfigure}[t]{\subfigwidth}
        \includegraphics[width=\linewidth]{img/exp_2/svm/2_3/max/train_matrix.png}
        \caption{Max, train}
    \end{subfigure}
    \hfill
    \begin{subfigure}[t]{\subfigwidth}
        \includegraphics[width=\linewidth]{img/exp_2/svm/2_3/max/test_boundary.png}
        \caption{Max, test}
    \end{subfigure}
    \hfill
    \begin{subfigure}[t]{\subfigwidth}
        \includegraphics[width=\linewidth]{img/exp_2/svm/2_3/max/test_matrix.png}
        \caption{Max, test}
    \end{subfigure} 
    
    \caption{Eksperyment 2 --- SVM na zbiorze 2\_3}
\end{figure}

\clearpage

\begin{figure}[H]\centering
    \begin{subfigure}[t]{\subfigwidth}
        \includegraphics[width=\linewidth]{img/exp_2/mlp/2_2/accuracy.png}
        \caption{Dokładność}
    \end{subfigure}
    \\
    \begin{subfigure}[t]{\subfigwidth}
        \includegraphics[width=\linewidth]{img/exp_2/mlp/2_2/min/train_boundary.png}
        \caption{Min, train}
    \end{subfigure}
    \hfill
    \begin{subfigure}[t]{\subfigwidth}
        \includegraphics[width=\linewidth]{img/exp_2/mlp/2_2/min/train_matrix.png}
        \caption{Min, train}
    \end{subfigure}
    \hfill
    \begin{subfigure}[t]{\subfigwidth}
        \includegraphics[width=\linewidth]{img/exp_2/mlp/2_2/min/test_boundary.png}
        \caption{Min, test}
    \end{subfigure}
    \hfill
    \begin{subfigure}[t]{\subfigwidth}
        \includegraphics[width=\linewidth]{img/exp_2/mlp/2_2/min/test_matrix.png}
        \caption{Min, test}
    \end{subfigure} 
    \\
    \begin{subfigure}[t]{\subfigwidth}
        \includegraphics[width=\linewidth]{img/exp_2/mlp/2_2/best/train_boundary.png}
        \caption{Best, train}
    \end{subfigure}
    \hfill
    \begin{subfigure}[t]{\subfigwidth}
        \includegraphics[width=\linewidth]{img/exp_2/mlp/2_2/best/train_matrix.png}
        \caption{Best, train}
    \end{subfigure}
    \hfill
    \begin{subfigure}[t]{\subfigwidth}
        \includegraphics[width=\linewidth]{img/exp_2/mlp/2_2/best/test_boundary.png}
        \caption{Best, test}
    \end{subfigure}
    \hfill
    \begin{subfigure}[t]{\subfigwidth}
        \includegraphics[width=\linewidth]{img/exp_2/mlp/2_2/best/test_matrix.png}
        \caption{Best, test}
    \end{subfigure} 
    \\
    \begin{subfigure}[t]{\subfigwidth}
        \includegraphics[width=\linewidth]{img/exp_2/mlp/2_2/max/train_boundary.png}
        \caption{Max, train}
    \end{subfigure}
    \hfill
    \begin{subfigure}[t]{\subfigwidth}
        \includegraphics[width=\linewidth]{img/exp_2/mlp/2_2/max/train_matrix.png}
        \caption{Max, train}
    \end{subfigure}
    \hfill
    \begin{subfigure}[t]{\subfigwidth}
        \includegraphics[width=\linewidth]{img/exp_2/mlp/2_2/max/test_boundary.png}
        \caption{Max, test}
    \end{subfigure}
    \hfill
    \begin{subfigure}[t]{\subfigwidth}
        \includegraphics[width=\linewidth]{img/exp_2/mlp/2_2/max/test_matrix.png}
        \caption{Max, test}
    \end{subfigure} 
    
    \caption{Eksperyment 2 --- MLP na zbiorze 2\_2}
\end{figure}

\begin{figure}[H]\centering
    \begin{subfigure}[t]{\subfigwidth}
        \includegraphics[width=\linewidth]{img/exp_2/mlp/2_3/accuracy.png}
        \caption{Dokładność}
    \end{subfigure}
    \\
    \begin{subfigure}[t]{\subfigwidth}
        \includegraphics[width=\linewidth]{img/exp_2/mlp/2_3/min/train_boundary.png}
        \caption{Min, train}
    \end{subfigure}
    \hfill
    \begin{subfigure}[t]{\subfigwidth}
        \includegraphics[width=\linewidth]{img/exp_2/mlp/2_3/min/train_matrix.png}
        \caption{Min, train}
    \end{subfigure}
    \hfill
    \begin{subfigure}[t]{\subfigwidth}
        \includegraphics[width=\linewidth]{img/exp_2/mlp/2_3/min/test_boundary.png}
        \caption{Min, test}
    \end{subfigure}
    \hfill
    \begin{subfigure}[t]{\subfigwidth}
        \includegraphics[width=\linewidth]{img/exp_2/mlp/2_3/min/test_matrix.png}
        \caption{Min, test}
    \end{subfigure} 
    \\
    \begin{subfigure}[t]{\subfigwidth}
        \includegraphics[width=\linewidth]{img/exp_2/mlp/2_3/best/train_boundary.png}
        \caption{Best, train}
    \end{subfigure}
    \hfill
    \begin{subfigure}[t]{\subfigwidth}
        \includegraphics[width=\linewidth]{img/exp_2/mlp/2_3/best/train_matrix.png}
        \caption{Best, train}
    \end{subfigure}
    \hfill
    \begin{subfigure}[t]{\subfigwidth}
        \includegraphics[width=\linewidth]{img/exp_2/mlp/2_3/best/test_boundary.png}
        \caption{Best, test}
    \end{subfigure}
    \hfill
    \begin{subfigure}[t]{\subfigwidth}
        \includegraphics[width=\linewidth]{img/exp_2/mlp/2_3/best/test_matrix.png}
        \caption{Best, test}
    \end{subfigure} 
    \\
    \begin{subfigure}[t]{\subfigwidth}
        \includegraphics[width=\linewidth]{img/exp_2/mlp/2_3/max/train_boundary.png}
        \caption{Max, train}
    \end{subfigure}
    \hfill
    \begin{subfigure}[t]{\subfigwidth}
        \includegraphics[width=\linewidth]{img/exp_2/mlp/2_3/max/train_matrix.png}
        \caption{Max, train}
    \end{subfigure}
    \hfill
    \begin{subfigure}[t]{\subfigwidth}
        \includegraphics[width=\linewidth]{img/exp_2/mlp/2_3/max/test_boundary.png}
        \caption{Max, test}
    \end{subfigure}
    \hfill
    \begin{subfigure}[t]{\subfigwidth}
        \includegraphics[width=\linewidth]{img/exp_2/mlp/2_3/max/test_matrix.png}
        \caption{Max, test}
    \end{subfigure} 
    
    \caption{Eksperyment 2 --- MLP na zbiorze 2\_3}
\end{figure}

\clearpage
% 5 strona --- Wyniki trzeciego eksperymentu dla dwóch sztucznie wygenerowanych zbiorów danych 2\_2 i 2\_3 oraz metody K-NN. Dla każdego zbioru należy pokazać wykres obrazujący zmianę wartości accuracy na zbiorach treningowym i testowym przy zmieniającym się parametrze n\_neighbours oraz wizualizacje przebiegu granicy decyzyjnej na zbiorach treningowym i testowym dla: najmniejszej, najlepszej (wartość accuracy na zbiorze testowym) i największej wartości tego parametru. Dodatkowo przy każdej wizualizacji należy pokazać jak wygląda macierz pomyłek.
\begin{figure}[H]\centering
    \begin{subfigure}[t]{\subfigwidth}
        \includegraphics[width=\linewidth]{img/exp_3/knn/2_2/accuracy.png}
        \caption{Dokładność}
    \end{subfigure}
    \\
    \begin{subfigure}[t]{\subfigwidth}
        \includegraphics[width=\linewidth]{img/exp_3/knn/2_2/min/train_boundary.png}
        \caption{Min, train}
    \end{subfigure}
    \hfill
    \begin{subfigure}[t]{\subfigwidth}
        \includegraphics[width=\linewidth]{img/exp_3/knn/2_2/min/train_matrix.png}
        \caption{Min, train}
    \end{subfigure}
    \hfill
    \begin{subfigure}[t]{\subfigwidth}
        \includegraphics[width=\linewidth]{img/exp_3/knn/2_2/min/test_boundary.png}
        \caption{Min, test}
    \end{subfigure}
    \hfill
    \begin{subfigure}[t]{\subfigwidth}
        \includegraphics[width=\linewidth]{img/exp_3/knn/2_2/min/test_matrix.png}
        \caption{Min, test}
    \end{subfigure} 
    \\
    \begin{subfigure}[t]{\subfigwidth}
        \includegraphics[width=\linewidth]{img/exp_3/knn/2_2/best/train_boundary.png}
        \caption{Best, train}
    \end{subfigure}
    \hfill
    \begin{subfigure}[t]{\subfigwidth}
        \includegraphics[width=\linewidth]{img/exp_3/knn/2_2/best/train_matrix.png}
        \caption{Best, train}
    \end{subfigure}
    \hfill
    \begin{subfigure}[t]{\subfigwidth}
        \includegraphics[width=\linewidth]{img/exp_3/knn/2_2/best/test_boundary.png}
        \caption{Best, test}
    \end{subfigure}
    \hfill
    \begin{subfigure}[t]{\subfigwidth}
        \includegraphics[width=\linewidth]{img/exp_3/knn/2_2/best/test_matrix.png}
        \caption{Best, test}
    \end{subfigure} 
    \\
    \begin{subfigure}[t]{\subfigwidth}
        \includegraphics[width=\linewidth]{img/exp_3/knn/2_2/max/train_boundary.png}
        \caption{Max, train}
    \end{subfigure}
    \hfill
    \begin{subfigure}[t]{\subfigwidth}
        \includegraphics[width=\linewidth]{img/exp_3/knn/2_2/max/train_matrix.png}
        \caption{Max, train}
    \end{subfigure}
    \hfill
    \begin{subfigure}[t]{\subfigwidth}
        \includegraphics[width=\linewidth]{img/exp_3/knn/2_2/max/test_boundary.png}
        \caption{Max, test}
    \end{subfigure}
    \hfill
    \begin{subfigure}[t]{\subfigwidth}
        \includegraphics[width=\linewidth]{img/exp_3/knn/2_2/max/test_matrix.png}
        \caption{Max, test}
    \end{subfigure} 
    
    \caption{Eksperyment 3 --- KNN na zbiorze 2\_2}
\end{figure}

\begin{figure}[H]\centering
    \begin{subfigure}[t]{\subfigwidth}
        \includegraphics[width=\linewidth]{img/exp_3/knn/2_3/accuracy.png}
        \caption{Dokładność}
    \end{subfigure}
    \\
    \begin{subfigure}[t]{\subfigwidth}
        \includegraphics[width=\linewidth]{img/exp_3/knn/2_3/min/train_boundary.png}
        \caption{Min, train}
    \end{subfigure}
    \hfill
    \begin{subfigure}[t]{\subfigwidth}
        \includegraphics[width=\linewidth]{img/exp_3/knn/2_3/min/train_matrix.png}
        \caption{Min, train}
    \end{subfigure}
    \hfill
    \begin{subfigure}[t]{\subfigwidth}
        \includegraphics[width=\linewidth]{img/exp_3/knn/2_3/min/test_boundary.png}
        \caption{Min, test}
    \end{subfigure}
    \hfill
    \begin{subfigure}[t]{\subfigwidth}
        \includegraphics[width=\linewidth]{img/exp_3/knn/2_3/min/test_matrix.png}
        \caption{Min, test}
    \end{subfigure} 
    \\
    \begin{subfigure}[t]{\subfigwidth}
        \includegraphics[width=\linewidth]{img/exp_3/knn/2_3/best/train_boundary.png}
        \caption{Best, train}
    \end{subfigure}
    \hfill
    \begin{subfigure}[t]{\subfigwidth}
        \includegraphics[width=\linewidth]{img/exp_3/knn/2_3/best/train_matrix.png}
        \caption{Best, train}
    \end{subfigure}
    \hfill
    \begin{subfigure}[t]{\subfigwidth}
        \includegraphics[width=\linewidth]{img/exp_3/knn/2_3/best/test_boundary.png}
        \caption{Best, test}
    \end{subfigure}
    \hfill
    \begin{subfigure}[t]{\subfigwidth}
        \includegraphics[width=\linewidth]{img/exp_3/knn/2_3/best/test_matrix.png}
        \caption{Best, test}
    \end{subfigure} 
    \\
    \begin{subfigure}[t]{\subfigwidth}
        \includegraphics[width=\linewidth]{img/exp_3/knn/2_3/max/train_boundary.png}
        \caption{Max, train}
    \end{subfigure}
    \hfill
    \begin{subfigure}[t]{\subfigwidth}
        \includegraphics[width=\linewidth]{img/exp_3/knn/2_3/max/train_matrix.png}
        \caption{Max, train}
    \end{subfigure}
    \hfill
    \begin{subfigure}[t]{\subfigwidth}
        \includegraphics[width=\linewidth]{img/exp_3/knn/2_3/max/test_boundary.png}
        \caption{Max, test}
    \end{subfigure}
    \hfill
    \begin{subfigure}[t]{\subfigwidth}
        \includegraphics[width=\linewidth]{img/exp_3/knn/2_3/max/test_matrix.png}
        \caption{Max, test}
    \end{subfigure} 
    
    \caption{Eksperyment 3 --- KNN na zbiorze 2\_3}
\end{figure}

\clearpage

% 6 strona --- Wyniki trzeciego eksperymentu dla dwóch sztucznie wygenerowanych zbiorów danych 2\_2 i 2\_3 oraz metody SVM. Dla każdego zbioru należy pokazać wykres obrazujący zmianę wartości accuracy na zbiorach treningowym i testowym przy zmieniającym się parametrze C oraz wizualizacje przebiegu granicy decyzyjnej na zbiorach treningowym i testowym dla: najmniejszej, najlepszej (wartość accuracy na zbiorze testowym) i największej wartości tego parametru. Dodatkowo przy każdej wizualizacji należy pokazać jak wygląda macierz pomyłek. Wartości parametru C powinny się zmieniać wykładniczo, a na wykresie dobrze jest zastosować skalę logarytmiczną.
\begin{figure}[H]\centering
    \begin{subfigure}[t]{\subfigwidth}
        \includegraphics[width=\linewidth]{img/exp_3/svm/2_2/accuracy.png}
        \caption{Dokładność}
    \end{subfigure}
    \\
    \begin{subfigure}[t]{\subfigwidth}
        \includegraphics[width=\linewidth]{img/exp_3/svm/2_2/min/train_boundary.png}
        \caption{Min, train}
    \end{subfigure}
    \hfill
    \begin{subfigure}[t]{\subfigwidth}
        \includegraphics[width=\linewidth]{img/exp_3/svm/2_2/min/train_matrix.png}
        \caption{Min, train}
    \end{subfigure}
    \hfill
    \begin{subfigure}[t]{\subfigwidth}
        \includegraphics[width=\linewidth]{img/exp_3/svm/2_2/min/test_boundary.png}
        \caption{Min, test}
    \end{subfigure}
    \hfill
    \begin{subfigure}[t]{\subfigwidth}
        \includegraphics[width=\linewidth]{img/exp_3/svm/2_2/min/test_matrix.png}
        \caption{Min, test}
    \end{subfigure}
    \\
    \begin{subfigure}[t]{\subfigwidth}
        \includegraphics[width=\linewidth]{img/exp_3/svm/2_2/best/train_boundary.png}
        \caption{Best, train}
    \end{subfigure}
    \hfill
    \begin{subfigure}[t]{\subfigwidth}
        \includegraphics[width=\linewidth]{img/exp_3/svm/2_2/best/train_matrix.png}
        \caption{Best, train}
    \end{subfigure}
    \hfill
    \begin{subfigure}[t]{\subfigwidth}
        \includegraphics[width=\linewidth]{img/exp_3/svm/2_2/best/test_boundary.png}
        \caption{Best, test}
    \end{subfigure}
    \hfill
    \begin{subfigure}[t]{\subfigwidth}
        \includegraphics[width=\linewidth]{img/exp_3/svm/2_2/best/test_matrix.png}
        \caption{Best, test}
    \end{subfigure}
    \\
    \begin{subfigure}[t]{\subfigwidth}
        \includegraphics[width=\linewidth]{img/exp_3/svm/2_2/max/train_boundary.png}
        \caption{Max, train}
    \end{subfigure}
    \hfill
    \begin{subfigure}[t]{\subfigwidth}
        \includegraphics[width=\linewidth]{img/exp_3/svm/2_2/max/train_matrix.png}
        \caption{Max, train}
    \end{subfigure}
    \hfill
    \begin{subfigure}[t]{\subfigwidth}
        \includegraphics[width=\linewidth]{img/exp_3/svm/2_2/max/test_boundary.png}
        \caption{Max, test}
    \end{subfigure}
    \hfill
    \begin{subfigure}[t]{\subfigwidth}
        \includegraphics[width=\linewidth]{img/exp_3/svm/2_2/max/test_matrix.png}
        \caption{Max, test}
    \end{subfigure}

    \caption{Eksperyment 3 --- SVM na zbiorze 2\_2}
\end{figure}

\begin{figure}[H]\centering
    \begin{subfigure}[t]{\subfigwidth}
        \includegraphics[width=\linewidth]{img/exp_3/svm/2_3/accuracy.png}
        \caption{Dokładność}
    \end{subfigure}
    \\
    \begin{subfigure}[t]{\subfigwidth}
        \includegraphics[width=\linewidth]{img/exp_3/svm/2_3/min/train_boundary.png}
        \caption{Min, train}
    \end{subfigure}
    \hfill
    \begin{subfigure}[t]{\subfigwidth}
        \includegraphics[width=\linewidth]{img/exp_3/svm/2_3/min/train_matrix.png}
        \caption{Min, train}
    \end{subfigure}
    \hfill
    \begin{subfigure}[t]{\subfigwidth}
        \includegraphics[width=\linewidth]{img/exp_3/svm/2_3/min/test_boundary.png}
        \caption{Min, test}
    \end{subfigure}
    \hfill
    \begin{subfigure}[t]{\subfigwidth}
        \includegraphics[width=\linewidth]{img/exp_3/svm/2_3/min/test_matrix.png}
        \caption{Min, test}
    \end{subfigure}
    \\
    \begin{subfigure}[t]{\subfigwidth}
        \includegraphics[width=\linewidth]{img/exp_3/svm/2_3/best/train_boundary.png}
        \caption{Best, train}
    \end{subfigure}
    \hfill
    \begin{subfigure}[t]{\subfigwidth}
        \includegraphics[width=\linewidth]{img/exp_3/svm/2_3/best/train_matrix.png}
        \caption{Best, train}
    \end{subfigure}
    \hfill
    \begin{subfigure}[t]{\subfigwidth}
        \includegraphics[width=\linewidth]{img/exp_3/svm/2_3/best/test_boundary.png}
        \caption{Best, test}
    \end{subfigure}
    \hfill
    \begin{subfigure}[t]{\subfigwidth}
        \includegraphics[width=\linewidth]{img/exp_3/svm/2_3/best/test_matrix.png}
        \caption{Best, test}
    \end{subfigure}
    \\
    \begin{subfigure}[t]{\subfigwidth}
        \includegraphics[width=\linewidth]{img/exp_3/svm/2_3/max/train_boundary.png}
        \caption{Max, train}
    \end{subfigure}
    \hfill
    \begin{subfigure}[t]{\subfigwidth}
        \includegraphics[width=\linewidth]{img/exp_3/svm/2_3/max/train_matrix.png}
        \caption{Max, train}
    \end{subfigure}
    \hfill
    \begin{subfigure}[t]{\subfigwidth}
        \includegraphics[width=\linewidth]{img/exp_3/svm/2_3/max/test_boundary.png}
        \caption{Max, test}
    \end{subfigure}
    \hfill
    \begin{subfigure}[t]{\subfigwidth}
        \includegraphics[width=\linewidth]{img/exp_3/svm/2_3/max/test_matrix.png}
        \caption{Max, test}
    \end{subfigure}

    \caption{Eksperyment 3 --- SVM na zbiorze 2\_3}
\end{figure}
\clearpage

%7 strona --- Wyniki trzeciego eksperymentu dla dwóch sztucznie wygenerowanych zbiorów danych 2\_2 i 2\_3 oraz sieci MLP. Dla każdego zbioru należy pokazać wykres obrazujący zmianę wartości accuracy na zbiorach treningowym i testowym przy zmieniającej się liczbie neuronów w warstwie ukrytej oraz wizualizacje przebiegu granicy decyzyjnej na zbiorach treningowym i testowym dla: najmniejszej, najlepszej (wartość accuracy na zbiorze testowym) i największej wartości tego parametru. Dodatkowo przy każdej wizualizacji należy pokazać jak wygląda macierz pomyłek.
\begin{figure}[H]\centering
    \begin{subfigure}[t]{\subfigwidth}
        \includegraphics[width=\linewidth]{img/exp_3/mlp/2_2/accuracy.png}
        \caption{Dokładność}
    \end{subfigure}
    \\
    \begin{subfigure}[t]{\subfigwidth}
        \includegraphics[width=\linewidth]{img/exp_3/mlp/2_2/min/train_boundary.png}
        \caption{Min, train}
    \end{subfigure}
    \hfill
    \begin{subfigure}[t]{\subfigwidth}
        \includegraphics[width=\linewidth]{img/exp_3/mlp/2_2/min/train_matrix.png}
        \caption{Min, train}
    \end{subfigure}
    \hfill
    \begin{subfigure}[t]{\subfigwidth}
        \includegraphics[width=\linewidth]{img/exp_3/mlp/2_2/min/test_boundary.png}
        \caption{Min, test}
    \end{subfigure}
    \hfill
    \begin{subfigure}[t]{\subfigwidth}
        \includegraphics[width=\linewidth]{img/exp_3/mlp/2_2/min/test_matrix.png}
        \caption{Min, test}
    \end{subfigure}
    \\
    \begin{subfigure}[t]{\subfigwidth}
        \includegraphics[width=\linewidth]{img/exp_3/mlp/2_2/best/train_boundary.png}
        \caption{Best, train}
    \end{subfigure}
    \hfill
    \begin{subfigure}[t]{\subfigwidth}
        \includegraphics[width=\linewidth]{img/exp_3/mlp/2_2/best/train_matrix.png}
        \caption{Best, train}
    \end{subfigure}
    \hfill
    \begin{subfigure}[t]{\subfigwidth}
        \includegraphics[width=\linewidth]{img/exp_3/mlp/2_2/best/test_boundary.png}
        \caption{Best, test}
    \end{subfigure}
    \hfill
    \begin{subfigure}[t]{\subfigwidth}
        \includegraphics[width=\linewidth]{img/exp_3/mlp/2_2/best/test_matrix.png}
        \caption{Best, test}
    \end{subfigure}
    \\
    \begin{subfigure}[t]{\subfigwidth}
        \includegraphics[width=\linewidth]{img/exp_3/mlp/2_2/max/train_boundary.png}
        \caption{Max, train}
    \end{subfigure}
    \hfill
    \begin{subfigure}[t]{\subfigwidth}
        \includegraphics[width=\linewidth]{img/exp_3/mlp/2_2/max/train_matrix.png}
        \caption{Max, train}
    \end{subfigure}
    \hfill
    \begin{subfigure}[t]{\subfigwidth}
        \includegraphics[width=\linewidth]{img/exp_3/mlp/2_2/max/test_boundary.png}
        \caption{Max, test}
    \end{subfigure}
    \hfill
    \begin{subfigure}[t]{\subfigwidth}
        \includegraphics[width=\linewidth]{img/exp_3/mlp/2_2/max/test_matrix.png}
        \caption{Max, test}
    \end{subfigure}

    \caption{Eksperyment 3 --- MLP na zbiorze 2\_2}
\end{figure}

\begin{figure}[H]\centering
    \begin{subfigure}[t]{\subfigwidth}
        \includegraphics[width=\linewidth]{img/exp_3/mlp/2_3/accuracy.png}
        \caption{Dokładność}
    \end{subfigure}
    \\
    \begin{subfigure}[t]{\subfigwidth}
        \includegraphics[width=\linewidth]{img/exp_3/mlp/2_3/min/train_boundary.png}
        \caption{Min, train}
    \end{subfigure}
    \hfill
    \begin{subfigure}[t]{\subfigwidth}
        \includegraphics[width=\linewidth]{img/exp_3/mlp/2_3/min/train_matrix.png}
        \caption{Min, train}
    \end{subfigure}
    \hfill
    \begin{subfigure}[t]{\subfigwidth}
        \includegraphics[width=\linewidth]{img/exp_3/mlp/2_3/min/test_boundary.png}
        \caption{Min, test}
    \end{subfigure}
    \hfill
    \begin{subfigure}[t]{\subfigwidth}
        \includegraphics[width=\linewidth]{img/exp_3/mlp/2_3/min/test_matrix.png}
        \caption{Min, test}
    \end{subfigure}
    \\
    \begin{subfigure}[t]{\subfigwidth}
        \includegraphics[width=\linewidth]{img/exp_3/mlp/2_3/best/train_boundary.png}
        \caption{Best, train}
    \end{subfigure}
    \hfill
    \begin{subfigure}[t]{\subfigwidth}
        \includegraphics[width=\linewidth]{img/exp_3/mlp/2_3/best/train_matrix.png}
        \caption{Best, train}
    \end{subfigure}
    \hfill
    \begin{subfigure}[t]{\subfigwidth}
        \includegraphics[width=\linewidth]{img/exp_3/mlp/2_3/best/test_boundary.png}
        \caption{Best, test}
    \end{subfigure}
    \hfill
    \begin{subfigure}[t]{\subfigwidth}
        \includegraphics[width=\linewidth]{img/exp_3/mlp/2_3/best/test_matrix.png}
        \caption{Best, test}
    \end{subfigure}
    \\
    \begin{subfigure}[t]{\subfigwidth}
        \includegraphics[width=\linewidth]{img/exp_3/mlp/2_3/max/train_boundary.png}
        \caption{Max, train}
    \end{subfigure}
    \hfill
    \begin{subfigure}[t]{\subfigwidth}
        \includegraphics[width=\linewidth]{img/exp_3/mlp/2_3/max/train_matrix.png}
        \caption{Max, train}
    \end{subfigure}
    \hfill
    \begin{subfigure}[t]{\subfigwidth}
        \includegraphics[width=\linewidth]{img/exp_3/mlp/2_3/max/test_boundary.png}
        \caption{Max, test}
    \end{subfigure}
    \hfill
    \begin{subfigure}[t]{\subfigwidth}
        \includegraphics[width=\linewidth]{img/exp_3/mlp/2_3/max/test_matrix.png}
        \caption{Max, test}
    \end{subfigure}

    \caption{Eksperyment 3 --- MLP na zbiorze 2\_3}
\end{figure}
\clearpage

% 8 strona --- Wyniki czwartego eksperymentu dla sztucznie wygenerowanego zbioru danych 2\_3 oraz sieci MLP. Dla rozważanego zbioru należy rozważyć przypadki z różną liczbą danych treningowych (parametr train\_size równy równy wartościom użytym odpowiednio w eksperymentach drugim i trzecim). Dla obu przypadków należy zaprezentować wykres zmian accuracy na zbiorach treningowym i testowym w kolejnych epokach oraz wizualizacje przebiegu granicy decyzyjnej na zbiorach treningowym i testowym dla epoki: zerowej (przed rozpoczęciem nauki), najlepszej (wartość accuracy na zbiorze testowym) i ostatniej (po zakończeniu nauki).
% Dodatkowo w każdym z przypadków należy uruchomić proces treningu 10 razy z różnymi wagami początkowymi i w tabeli zamieścić wartości accuracy na zbiorze testowym i treningowym dla epoki: pierwszej (początek nauki), najlepszej (wartość accuracy na zbiorze testowym) i ostatniej (po zakończeniu nauki). W przypadku wartości najlepszej należy również podać numer epoki kiedy ją osiągnięto. Liczbę neuronów w warstwie ukrytej należy dobrać jako tą optymalną wynikającą odpowiednio z eksperymentów drugiego i trzeciego.

\renewcommand*{\subfigwidth}{0.24\textwidth}

\vspace*{-6em}
\begin{figure}[H]\centering
    \begin{subfigure}[t]{\subfigwidth}
        \includegraphics[width=\linewidth]{img/exp_4/set_0.8/accuracies.png}
        \caption{Dokładność}
    \end{subfigure}
    \hfill
    \begin{subfigure}[t]{\subfigwidth}
        \includegraphics[width=\linewidth]{img/exp_4/set_0.8/first/train_boundary.png}
        \caption{Pierwszy, train set}
    \end{subfigure}
    \hfill
    \begin{subfigure}[t]{\subfigwidth}
        \includegraphics[width=\linewidth]{img/exp_4/set_0.8/best/train_boundary.png}
        \caption{Najlepszy, train set}
    \end{subfigure}
    \hfill
    \begin{subfigure}[t]{\subfigwidth}
        \includegraphics[width=\linewidth]{img/exp_4/set_0.8/last/train_boundary.png}
        \caption{Ostatni, train set}
    \end{subfigure}
    \\
    \hspace{\subfigwidth}
    \hfill
    \begin{subfigure}[t]{\subfigwidth}
        \includegraphics[width=\linewidth]{img/exp_4/set_0.8/first/test_boundary.png}
        \caption{Pierwszy, test set}
    \end{subfigure}
    \hfill
    \begin{subfigure}[t]{\subfigwidth}
        \includegraphics[width=\linewidth]{img/exp_4/set_0.8/best/test_boundary.png}
        \caption{Najlepszy, test set}
    \end{subfigure}
    \hfill
    \begin{subfigure}[t]{\subfigwidth}
        \includegraphics[width=\linewidth]{img/exp_4/set_0.8/last/test_boundary.png}
        \caption{Ostatni, test set}
    \end{subfigure}
    \caption{Trenowanie na 0.8 danych}
\end{figure}

\begin{figure}[H]\centering
    \begin{subfigure}[t]{\subfigwidth}
        \includegraphics[width=\linewidth]{img/exp_4/set_0.2/accuracies.png}
        \caption{Dokładność}
    \end{subfigure}
    \hfill
    \begin{subfigure}[t]{\subfigwidth}
        \includegraphics[width=\linewidth]{img/exp_4/set_0.2/first/train_boundary.png}
        \caption{Pierwszy, train set}
    \end{subfigure}
    \hfill
    \begin{subfigure}[t]{\subfigwidth}
        \includegraphics[width=\linewidth]{img/exp_4/set_0.2/best/train_boundary.png}
        \caption{Najlepszy, train set}
    \end{subfigure}
    \hfill
    \begin{subfigure}[t]{\subfigwidth}
        \includegraphics[width=\linewidth]{img/exp_4/set_0.2/last/train_boundary.png}
        \caption{Ostatni, train set}
    \end{subfigure}
    \\
    \hspace{\subfigwidth}
    \hfill
    \begin{subfigure}[t]{\subfigwidth}
        \includegraphics[width=\linewidth]{img/exp_4/set_0.2/first/test_boundary.png}
        \caption{Pierwszy, test set}
    \end{subfigure}
    \hfill
    \begin{subfigure}[t]{\subfigwidth}
        \includegraphics[width=\linewidth]{img/exp_4/set_0.2/best/test_boundary.png}
        \caption{Najlepszy, test set}
    \end{subfigure}
    \hfill
    \begin{subfigure}[t]{\subfigwidth}
        \includegraphics[width=\linewidth]{img/exp_4/set_0.2/last/test_boundary.png}
        \caption{Ostatni, test set}
    \end{subfigure}
    \caption{Trenowanie na 0.2 danych}
\end{figure}

\begin{table}[H]
    \tiny
    \pgfplotstabletypesetfile[
        columns/seed/.style                 = {precision=0},
        columns/train_accuracy_first/.style = {column name = {pierwszy, train}},
        columns/train_accuracy_best/.style   = {column name = {najlepszy, train}},
        columns/train_best_epoch/.style      = {column name = {najlepsza epoka}, precision=0},
        columns/train_accuracy_end/.style   = {column name = {ostatni, train}},
        columns/test_accuracy_first/.style  = {column name = {pierwszy, test}},
        columns/test_accuracy_best/.style   = {column name = {najlepszy, test}},
        columns/test_best_epoch/.style      = {column name = {najlepsza epoka}, precision = 0},
        columns/test_accuracy_end/.style    = {column name = {ostatni, test}},
        dec sep align,
        fixed, precision=3, zerofill
    ]{img/exp_4/set_0.8/different_seed_stats.dat}
    \caption{Trenowanie na 0.8 danych}
\end{table}

\begin{table}[H]
    \tiny
    \pgfplotstabletypesetfile[
        columns/seed/.style                 = {precision=0},
        columns/train_accuracy_first/.style = {column name = {pierwszy, train}},
        columns/train_accuracy_best/.style   = {column name = {najlepszy, train}},
        columns/train_best_epoch/.style      = {column name = {najlepsza epoka}, precision=0},
        columns/train_accuracy_end/.style   = {column name = {ostatni, train}},
        columns/test_accuracy_first/.style  = {column name = {pierwszy, test}},
        columns/test_accuracy_best/.style   = {column name = {najlepszy, test}},
        columns/test_best_epoch/.style      = {column name = {najlepsza epoka}, precision = 0},
        columns/test_accuracy_end/.style    = {column name = {ostatni, test}},
        dec sep align,
        fixed, precision=3, zerofill
    ]{img/exp_4/set_0.2/different_seed_stats.dat}
    \caption{Trenowanie na 0.2 danych}
\end{table}


\clearpage

9 strona --- Opis wniosków z eksperymentów przeprowadzonych na sztucznie wygenerowanych zbiorach.
W przypadku wszystkich ekseprymentów należy zwrócić uwagę na kształt uzyskiwanych granic decyzyjnych 
i związane z nim zdolności uogólniajace poszczególnych rodzajów klasyfikatorów (wpływ hiperparametrów) 
oraz wpływ liczby danych treningowych. W eksperymencie czwartym należy dodatkowo skupić się na zdolnościach 
uogólniających w kolejnych epokach nauki oraz na wpływie sposobu zaincjalizowania sieci.
Wnioski powinny mieć charakter ogólny, pozwalający przenieść je na przypadek, w którym nie ma możliwości
zwizualizowania danych. Każdy wniosek powinien być poparty odniesieniami do wyników przedstawionych 
na pierwszych czterech stronach raportu.
\subsection*{Eksperyment 1}
Z eksperymentów na wygenerowanych danych można wyciągnąć wnioski dotyczące 
elastyczności rozważanych klasyfikatorów oraz wpływu ich hiperparametrów na 
kształt granic decyzyjnych. W przypadku klasyfikatora SVM istotnym parametrem 
jest kernel, który definiuje kształt granicy decyzyjnej. Porównując różne kernele, 
takie jak linear i rbf, można zauważyć różnice w elastyczności modelu. 
Podobnie, dla klasyfikatora opartego o sieć MLP, parametr activation wpływa na 
kształt granicy decyzyjnej, dlatego porównanie różnych funkcji aktywacji, 
takich jak identity i relu, jest istotne dla oceny elastyczności modelu.
\subsection*{Eksperyment 2}
Drugi eksperyment koncentruje się na wpływie elastyczności modelu na 
jego zdolności generalizacyjne. Zmiana hiperparametrów, takich jak liczba 
sąsiadów dla klasyfikatora K-NN, parametr C dla klasyfikatora SVM oraz liczba 
neuronów w warstwie ukrytej dla sieci MLP, ma istotny wpływ na zdolności 
generalizacyjne modelu. Badanie tego wpływu na podziale danych na zbiór treningowy 
i testowy pozwala ocenić skuteczność klasyfikacji na nowych, 
nie widzianych wcześniej danych, co można zaobserować na przeprowadzonych testach.
\subsection*{Eksperyment 3}
Trzeci eksperyment rozważa wpływ zmiany hiperparametrów modelu na jego zdolności 
generalizacyjne w przypadku niewielkiej liczby danych treningowych równej 0.2.
Przy ograniczonym zbiorze treningowym istotne staje się dobranie optymalnych hiperparametrów, 
aby uniknąć przeuczenia, co można zaobserwować na zmianie wykresu accuracy, 
gdzie w przypadku zbioru \texttt{2\_2} przez pierwsze 30 ukrytych warstw 
nie ma zmiany, a w przypadku \texttt{2\_3} ma to ogromne znaczenie, dlatego przedstawiono tylko pierwsze warstwy.
Wynika to ze specyfikacji danego zbioru.
Porównanie różnych kombinacji hiperparametrów pozwala zidentyfikować te, które prowadzą do najlepszych wyników 
przy ograniczonej liczbie danych.
\subsection*{Eksperyment 4}
Czwarty eksperyment skupia się na obserwacji procesu nauki sieci MLP. 
Dzięki wykorzystaniu metody \texttt{partial\_fit} 
można śledzić zmiany skuteczności modelu na zbiorze treningowym i 
testowym po kolejnych epokach nauki. 
Dodatkowo, badając wpływ różnych inicjalizacji wag neuronów na finalne 
rezultaty klasyfikacji, można lepiej 
zrozumieć proces uczenia się sieci i jego wpływ na zdolności generalizacyjne modelu.

\clearpage

10 strona --- Opis działania analizowanych metod klasyfikacji w przypadku rzeczywistych zbiorów dnaych. 
Podczas tworzenia klasyfikatorów warto skorzystać z wniosków wyciągniętych podczas wcześniejszych eksperymentów. 
Uzyskane wyniki należy zaprezentować w zwartej formie (warto wykorzystać tabele i/lub wykresy), 
a wnioski należy poprzeć odwołaniami do tych wyników.

Rysunki 15 - 17 przedstawiają wyniki eksperymentów z użyciem KNN, SVM i MLP na zbiorach Iris, Wine oraz Breast Cancer Wisconsin
W przypadku metody K-NN skoncentrowaliśmy się na optymalnej liczbie sąsiadów \(n_{\text{neighbours}}\). Dla metody SVM dobraliśmy optymalną 
wartość parametru C, kontrolującego tolerancję naruszenia marginesu. Natomiast dla sieci MLP ustaliliśmy optymalną liczbę neuronów 
w warstwie ukrytej.

Wyniki działania klasyfikatorów na rzeczywistych danych zostały przedstawione w formie tabel oraz wykresów. 
Dokładnie przeanalizowaliśmy 
skuteczność klasyfikacji na zbiorze treningowym i testowym, a także zbadaliśmy macierz pomyłek dla każdej z metod. Nasza analiza wskazuje, 
że wybrane metody klasyfikacji dobrze radzą sobie z rzeczywistymi danymi. Klasyfikatory osiągają zadowalające wyniki zarówno na zbiorach 
treningowych, jak i testowych


\begin{figure}[H]\centering
    \begin{subfigure}[t]{\subfigwidth}
        \includegraphics[width=\linewidth]{img/other_datasets/knn/iris_accuracy.png}
        \caption{KNN, accuracy}
    \end{subfigure}
    \hfill
    \begin{subfigure}[t]{\subfigwidth}
        \includegraphics[width=\linewidth]{img/other_datasets/svm/iris_accuracy.png}
        \caption{SVM, accuracy}
    \end{subfigure}
    \hfill
    \begin{subfigure}[t]{\subfigwidth}
        \includegraphics[width=\linewidth]{img/other_datasets/mlp/iris_accuracy.png}
        \caption{MLP, accuracy}
    \end{subfigure}
    \hfill
    \begin{subfigure}[t]{\subfigwidth}
        \includegraphics[width=\linewidth]{img/other_datasets/mlp/iris_best_boundary_test.png}
        \caption{Best case}
    \end{subfigure}
    \caption{Wyniki eksperymentów na zbiorze Iris}
\end{figure}

\begin{figure}[H]\centering
    \begin{subfigure}[t]{\subfigwidth}
        \includegraphics[width=\linewidth]{img/other_datasets/knn/wine_accuracy.png}
        \caption{KNN, accuracy}
    \end{subfigure}
    \hfill
    \begin{subfigure}[t]{\subfigwidth}
        \includegraphics[width=\linewidth]{img/other_datasets/svm/wine_accuracy.png}
        \caption{SVM, accuracy}
    \end{subfigure}
    \hfill
    \begin{subfigure}[t]{\subfigwidth}
        \includegraphics[width=\linewidth]{img/other_datasets/mlp/wine_accuracy.png}
        \caption{MLP - accuracy}
    \end{subfigure}
    \hfill
    \begin{subfigure}[t]{\subfigwidth}
        \includegraphics[width=\linewidth]{img/other_datasets/mlp/wine_best_boundary_test.png}
        \caption{Best case}
    \end{subfigure}
    \caption{Wyniki eksperymentów na zbiorze Wine}
\end{figure}

\begin{figure}[H]\centering
    \begin{subfigure}[t]{\subfigwidth}
        \includegraphics[width=\linewidth]{img/other_datasets/knn/breast_accuracy.png}
        \caption{KNN - accuracy}
    \end{subfigure}
    \hfill
    \begin{subfigure}[t]{\subfigwidth}
        \includegraphics[width=\linewidth]{img/other_datasets/svm/breast_accuracy.png}
        \caption{SVM - accuracy}
    \end{subfigure}
    \hfill
    \begin{subfigure}[t]{\subfigwidth}
        \includegraphics[width=\linewidth]{img/other_datasets/mlp/breast_accuracy.png}
        \caption{MLP - accuracy}
    \end{subfigure}
    \hfill
    \begin{subfigure}[t]{\subfigwidth}
        \includegraphics[width=\linewidth]{img/other_datasets/mlp/breast_best_boundary_test.png}
        \caption{Best case}
    \end{subfigure}
    \caption{Wyniki eksperymentów na zbiorze Breast Cancer Wisconsin}
\end{figure}



\end{document}