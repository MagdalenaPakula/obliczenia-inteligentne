\documentclass[conference]{IEEEtran}
\IEEEoverridecommandlockouts
\usepackage{cite}
\usepackage{amsmath,amssymb,amsfonts}
\usepackage{graphicx}
\usepackage{textcomp}
\usepackage{xcolor}

\begin{document}

\title{Report 3 - Obliczenia Inteligentne}
\author{
    \IEEEauthorblockN{Magdalena Pakuła}
    \IEEEauthorblockA{WFTIMS\\
    Politechnika Łódzka\
    Email: 254220@edu.p.lodz.pl}
    \and
    \IEEEauthorblockN{Jakub Pawlak}
    \IEEEauthorblockA{WFTIMS\\
    Politechnika Łódzka\\
    Email: 234767@edu.p.lodz.pl}
}


\maketitle

\begin{abstract}
This document provides guidelines and a template for writing conference papers using \LaTeX\ in accordance with IEEE standards. It covers various aspects such as formatting, styling, and referencing.
\end{abstract}

\begin{IEEEkeywords}
component, formatting, style, styling
\end{IEEEkeywords}

\section{Introduction}
This document serves as a model and instructions for formatting conference papers using \LaTeX\ according to IEEE standards.

\section{Ease of Use}

\subsection{Maintaining the Integrity of the Specifications}
The IEEEtran class file is utilized to format and style the text of your paper. Please adhere to the prescribed margins, column widths, line spaces, and text fonts.

\section{Prepare Your Paper Before Styling}
Before formatting your paper, ensure all content is written and organized. Please review sections \ref{AA}--\ref{SCM} for proofreading and grammar tips.

\subsection{Abbreviations and Acronyms}\label{AA}
Define abbreviations when first used in the text. Abbreviations such as IEEE, SI, MKS, CGS, etc., need not be defined again.

\subsection{Units}
Follow SI units primarily, with English units in parentheses if necessary. Ensure consistency in unit usage and avoid mixing SI and CGS units.

\subsection{Equations}
Number equations consecutively and ensure symbols are defined before or immediately following the equation.

\subsection{\LaTeX-Specific Advice}
Use soft cross-references (\verb|\eqref|) and avoid \verb|{eqnarray}| environment. Use \verb|{align}| or \verb|{IEEEeqnarray}| instead.

\subsection{Some Common Mistakes}\label{SCM}
Avoid common mistakes such as incorrect punctuation, confusing homophones, and misuse of abbreviations.

\subsection{Authors and Affiliations}
List authors and affiliations sequentially. Keep affiliations concise.

\subsection{Identify the Headings}
Organize headings as component heads and text heads. Follow the hierarchical structure for text heads.

\subsection{Figures and Tables}
Position figures and tables appropriately and ensure proper labeling. Use descriptive labels and avoid ambiguity.

\section*{Acknowledgment}
Acknowledgments should be brief and located at the end of the text.

\section*{References}
Number citations consecutively within brackets and ensure proper punctuation.

\begin{thebibliography}{00}
\bibitem{b1} G. Eason, B. Noble, and I. N. Sneddon, ``On certain integrals of Lipschitz-Hankel type involving products of Bessel functions,'' Phil. Trans. Roy. Soc. London, vol. A247, pp. 529--551, April 1955.
% Include other references here
\end{thebibliography}

\end{document}
